\documentclass[12pt,a4paper]{article}

\begin{document}

\makeatletter
\newenvironment{owndescription}
               {\list{}{\labelwidth\z@ \itemindent-\leftmargin
                        \let\makelabel\owndescriptionlabel}}
               {\endlist}
\newcommand*\owndescriptionlabel[1]{\hspace\labelsep
  \normalfont #1}
\makeatother

\title{Using \texttt{tabudes} with \texttt{Discreta}}

\author{Kari J. Nurmela\\
Department of Computer Science and Engineering\\
Helsinki University of Technology\\
P.O.\ Box 5400\\
FIN--02015 HUT\\
Finland\\e-mail: \texttt{Kari.Nurmela@hut.fi}}

\date{\today}

\maketitle

\section{Introduction}
\label{sec:intro}

This document describes how to search combinatorial designs with
with the help of \texttt{tabudes} and \texttt{Discreta}.

\section{Using the program}

\newcommand{\Var}[1]{\textbf{#1}}
\newcommand{\Val}[1]{\textit{#1}}
\newcommand{\NIY}{\textit{(not implemented yet)}}

The \texttt{tabudes} program is invoked from command line with the
following syntax:
\begin{quote}
  \Var{tabudes}\ \Val{variable1}\ \Val{value1}\ \Val{variable2}\
  \Val{value2}\ \ldots
\end{quote}
Each variable is followed by its value. We now describe each variable
and how they define the search that the program performs. The variable
names and keywords recognized by the program are set in \Var{bold},
the parameters (integers if not otherwise stated) are written in
\Val{italics}.

\subsection{Common variables}

\begin{owndescription}
\item[\Var{discretafile}] The file where the Kramer-Mesner matrix (and
  relevant information) is stored by \texttt{Discreta}. This option
  must always be specified.
\item[\Var{algo}] The search algorithm. Must be specified and can be
  one of the following:
  \begin{owndescription}
  \item[\Val{tabu}] Tabu search. See section~\ref{sec:tabu}.
  \item[\Val{SA}] Simulated annealing. See section~\ref{sec:sa}.
  \item[\Val{deluge}] Great deluge. See section~\ref{sec:deluge}.
  \item[\Val{record}] Record-to-record travel. See
    section~\ref{sec:record}.
  \item[\Val{threshold}] Threshold accepting. See
    section~\ref{sec:threshold}.
  \item[\Val{local}] Local optimization. Makes penalty-improving moves
    until no such move exists. See section~\ref{sec:local}.
  \item[\Val{iterpen}] Iterate within penalty limits. See
    section~\ref{sec:iterpen}.
  \item[\Val{rwalk}] Random walk. See section~\ref{sec:rwalk}.
  \item[\Val{rwalkpen}] Random walk within penalty limits. See
    section~\ref{sec:rwalkpen}.
  \item[\Val{none}] No search is performed.
  \end{owndescription}
\item[\Var{check}] If \Val{true}, and a solution satisfying the
  constraints (\Var{minpen}, \Var{maxpen}, \Var{minvol}, \Var{maxvol})
  is achieved, then the integrity, penalty, and
  volume of the solution are checked before printing (or saving) the
  solution. Default is true.
\item[\Var{maxpen}] The maximum acceptable penalty for a solution to
  be shown and/or saved to a file. Default maximum value of the
  penalty data type.
\item[\Var{minpen}] The minimum acceptable penalty for a solution to
  be shown and/or saved to a file. Default minimum value of the
  penalty data type.
\item[\Var{maxvol}] The maximum acceptable volume for a solution to
  be shown and/or saved to a file. Default maximum value of the
  volume data type.
\item[\Var{minvol}] The minimum acceptable volume for a solution to
  be shown and/or saved to a file. Default minimum value of the
  volume data type.
\item[\Var{discretapenalty}] If the final solution has penalty less
  than or equal to \Var{discretapenalty}, then the solution vector (in
  a format that can be read back to \texttt{Discreta}) is written in
  file \Var{discretaoutput}, if given.
\item[\Var{discretaoutput}] The name of the file where the final
  solution (if penalty and volume limits are fulfilled) is saved in
  a format that can be read by Discreta.
\item[\Var{outfile}] Optional name of the file where the final solution is
  saved (in a format that can be later read again to \texttt{tabudes})
  if it fulfills the penalty and volume limits (\Var{maxpen},
  \Var{minpen}, \Var{maxvol}, \Var{minvol}).
\item[\Var{verbose}] Controls the verbosity of the search process: 0
  prints only a little of information of the progress of the search,
  larger values print more. Default 0.
\item[\Var{seed}] The seed for the pseudorandom number generator. If
  not given, then a seed value is taken from the system clock.
\item[\Var{coverpenalty}] Determines the penalty function. Can be one
  of the following:
  \begin{owndescription}
  \item[\Val{design}] Default. If an orbit is covered $n$ times, and
    it should be covered $\lambda$ times, penalty is increased by
    $\left| n - \lambda \right|$.
  \item[\Val{cover}] The same as \Val{design}, except that only orbits
    that are covered too few times are considered.
  \item[\Val{pack}] The same as \Val{design}, except that only orbits
    that are covered too many times are considered.
  \item[\Val{asymhigh}] The same as \Val{design}, except that each orbit
    that is covered too many times adds another \Var{asymadd} to the
    penalty.
  \item[\Val{asymlow}] The same as \Val{design}, except that each orbit
    that is covered too few times adds another \Var{asymadd} to the
    penalty.
  \end{owndescription}
\item[\Var{asymadd}] The number that is added to the penalty for each
  $t$-set whose coverage is less or more than $\lambda$ when
  \Var{coverpenalty} is \Val{asymlow} or \Val{asymhigh}, respectively.
\item[\Var{lambda}] The number the orbits of $t$-sets are to be
  covered (upper bound for packings, lower bound for coverings).
\item[\Var{order}] If \Var{discretafile} does not contain the ``\%
  order:'' line in the preamble, the order of the symmetry group can
  be given on the command line.
\end{owndescription}

\subsection{Algorithm parameters}

Depending on the search algorithm used, there are several parameters
that can be used.

\subsubsection{Tabu search}
\label{sec:tabu}

Tabu search parameters define the type of tabu list, the attributes
used, and so on.

\begin{owndescription}
\item[\Var{attr}] Determines the attributes used for the tabu list.
  The value of \Var{attr} is a white space separated string which is
  interpreted in postfix notation according the following patterns:
  \begin{owndescription}
  \item[\Var{remove}] Push the following condition on the stack: if an
    orbit is added to the solution, then the condition forbids the
    removal of the same orbit.
  \item[\Var{add}] Push the following condition on the stack: if an
    orbit is removed from the solution, then the condition forbids the
    addition of the same orbit.
  \item[\Var{addall}] Push the following condition on the stack: if a
    set of orbitss is removed from the solution, then the condition
    forbids the addition of exactly the same set of orbits.
  \item[\Var{index}] Push the following condition on the stack: if the
    orbit at position $i$ in the solution vector is changed, the
    condition prevents the change of the same position.
  \item[\Var{change}] Push the following condition on the stack: if a
    move removes orbit $a$ and adds orbit $b$, then another move
    cannot remove $b$ and add $a$.
  \item[\Val{loop} \Var{penvol}] Push the following condition on the
    stack: the last \Val{loop} penalties and volumes on the
    search path are recorded. If the next move would result in an
    identical sequence of penalties and volumes, it is forbidden.
  \item[\Val{n} \Var{and}] Pop \Val{n} conditions from the stack. Push
    a condition that composes the popped conditions.
  \item[\Val{l} \Var{tl}] Pop a condition from the stack. Add the
    condition with constant tabu tenure \Val{l} to the tabu list
    structure.
  \item[\Val{l} \Val{n} \Var{tls}] Pop \Val{n} conditions from the
    stack. Add the conditions with constant tabu tenure \Val{l} to the
    tabu list structure.
  \item[\Val{wl} \Var{dt} \Val{tl1} \Val{tl2}\ldots \Val{tlm}
    \Var{dtl}] Pop a condition from the stack. Add the condition to
    the tabu list structure with a dynamic tabu tenure: if the cost in
    the two latest successive time windows of length \Val{wl} are
    similar, then the tabu
    tenure is changed to the next tenure in the list.
  \item[\Val{cc} \Var{ct} \Val{tl1} \Val{tl2}\ldots \Val{tlm}
    \Var{ctl}] Pop a condition from the stack. Add the condition to
    the tabu list structure with a cyclic dynamic tabu tenure: after
    each \Val{cc} iterations the tabu tenure is changed to the next
    tenure on the list.
  \item[\Var{neigh}] Enables the special condition(s) generated by the
    neighborhood(s). (Not currently used.)
  \end{owndescription}
  In the beginning the stack is empty. After interpreting the value
  string of \Var{attr} the stack should also be empty and a number of
  attribute/tenure pairs should have been added to the tabu list
  structure.
\item[\Var{maxiter}] The maximum number of moves made during the
  search.
\item[\Var{endlimit}] If a solution is found with penalty at most
  \Var{endlimit}, the search is terminated.
\item[\Var{largeremaining}] If more than one neighborhood is given:
  the search is normally done using only the first neighborhood.
  However, when the current penalty equals to the best penalty found
  during this search process, then at most \Var{largeremaining} moves
  are made using all the neighborhoods. If equally good moves are
  found, those in an earlier neighborhood in the neighborhood list are
  preferred. See also \Var{smallpenalty}.
\item[\Var{smallpenalty}] If the penalty is larger than
  \Var{smallpenalty}, then only the first neighborhood is used, even
  if the current penalty equals to the best penalty found during this
  search process.
\end{owndescription}

\subsubsection{Simulated annealing}
\label{sec:sa}

\begin{owndescription}
\item[\Var{IT}] Initial temperature.
\item[\Var{IP}] Tries to find approximate initial temperature such
  that the probability of accepting a random move in the initial
  solution is \Var{IP}.
\item[\Var{L}] The number of iterations at each temperature.
\item[\Var{LF}] The number of iterations at each temperature can be
  alternatively specified by \Var{LF}: the number of iterations is the
  size of the neighborhood multiplied by \Var{LF}.
\item[\Var{CF}] The temperature is reduced by multiplying the old
  temperature by \Var{CF}.
\item[\Var{frozen}] The search is terminated when \Var{frozen}
  successive temperatures have passed without a single
  penalty-changing move.
\item[\Var{endlimit}] If a solution is found with penalty at most
  \Var{endlimit}, the search is terminated.
\end{owndescription}

\subsubsection{Great deluge}
\label{sec:deluge}

\begin{owndescription}
\item[\Var{frozen}] If no change in penalty (while the water level
  equals the penalty) during \Var{frozen} candidate moves, the search
  is terminated, default 1000.
\item[\Var{down}] The amount of water level change after each penalty
  change calculation, default 0.001.
\item[\Var{initlevel}] The initial water level, default is the penalty
  of the initial solution.
\item[\Var{endlimit}] If a solution is found with penalty at most
  \Var{endlimit}, the search is terminated.
\end{owndescription}

\subsubsection{Record-to-record travel}
\label{sec:record}

\begin{owndescription}
\item[\Var{frozen}] If no new record has been found during
  \Var{frozen} candidate moves, the search is terminated, default
  1000.
\item[\Var{deviation}] Moves which result in a solution whose penalty
  deviates from the current record at most by \Var{deviation} are
  accepted.
\item[\Var{endlimit}] If a solution is found with penalty at most
  \Var{endlimit}, the search is terminated.
\item[\Var{specialrecord}] If not given, the neighborhood is
  initialized after each accepted move. If \Var{specialrecord} is
  given, the neighborhood is initialized after \Var{specialrecord}
  candidate moves. Useful for example when using neighborhoods like
  \Var{correct}.
\end{owndescription}

\subsubsection{Threshold accepting}
\label{sec:threshold}

\begin{owndescription}
\item[\Var{frozen}] The search is terminated when \Var{frozen}
  successive temperatures have passed without a single
  penalty-changing move.
\item[\Var{threshold}] Initial threshold.
\item[\Var{L}] If specified, \Var{L} moves are tried each time before
  changing the threshold.
\item[\Var{LF}] If specified, \Var{LF} times the neighborhood size
  moves are tried each time before changing the threshold.
\item[\Var{down}] The decrement of the threshold.
\item[\Var{endlimit}] If a solution is found with penalty at most
  \Var{endlimit}, the search is terminated.
\end{owndescription}

\subsubsection{Iterate within penalty limits}
\label{sec:iterpen}

This algorithm makes random moves by finding all the moves in the
current neighborhood that satisfy the penalty limits, and then by
taking one of them until no such moves can be found or \Var{maxiter}
moves have been made.

\begin{owndescription}
\item[\Var{maxiter}] At most \Var{maxiter} moves are made.
\item[\Var{minpen}] The smallest acceptable penalty.
\item[\Var{maxpen}] The largest acceptable penalty.
\end{owndescription}

\subsubsection{Random walk}
\label{sec:rwalk}

Makes random moves.

\begin{owndescription}
\item[\Var{maxiter}] The walk is terminated when the number of moves
  made reaches \Var{maxiter}.
\end{owndescription}

\subsubsection{Random walk within penalty limits}
\label{sec:rwalkpen}

This algorithm tries random moves and accepts those that satisfy the
penalty limits. The iteration is continued until no such moves can be
found or \Var{maxiter} moves have been made.

\begin{owndescription}
\item[\Var{maxiter}] At most \Var{maxiter} moves are made.
\item[\Var{minpen}] The smallest acceptable penalty.
\item[\Var{maxpen}] The largest acceptable penalty.
\end{owndescription}

\subsubsection{Local optimization}
\label{sec:local}

\subsection{Initial solution}

\begin{owndescription}
\item[\Var{initsol}] Specifies the type of the initial solution. Can
  be one of the following (default is \Val{atleastvol}):
  \begin{owndescription}
  \item[\Val{atleastvol}] Orbits are added to the solution until the
    volume is \Var{vol} or more.
  \item[\Val{atmostvol}] Orbits are added until the next random patch
    would result exceeding the volume specified by \Var{vol}.
  \item[\Val{atmostpen}] Orbits are added until no more patches can
    be added without exceeding the penalty specified by \Var{pen}.
  \item[\Val{read}] Read the initial solution from the file specified
    by \Var{infile}. (The file is written earlier using
    \Var{outfile}.)
  \item[\Val{coverage}] Make initial solution as follows: start with
    an empty solution and add random orbits until the total coverage
    is equal to or greater than \Var{lambda} times the number of
    orbits of $t$-sets.
  \item[\Val{volumes}] Pick random orbits according to parameter
    \Var{volcounts}.
  \item[\Val{empty}] Start the search with empty initial solution.
  \end{owndescription}
\item[\Var{vol}] See owndescription of
  \Var{initsol}$\in\{$\Val{atleastvol}, \Val{atmostvol}$\}$ above.
\item[\Var{infile}] The file where the initial solution is read from
  when \Var{initsol}$=$\Val{read}.
\item[\Var{partsize}] See owndescription of \Val{initsol}=\Val{partition}
  above (and owndescription of \Var{partition} in
  Section~\ref{sec:neighborhood} below).
\item[\Var{volcounts}] The value of \Var{volcounts} is a
  space-separated string of form ``$v_1$ $c_1$ $v_2$ $c_2$ \ldots
  $v_n$ $c_n$''. The initial solution (\Var{initsol}$=$\Val{volumes})
  is formed by selecting $c_1$ orbits of volume $v_1$, $c_2$ orbits of
  volume $v_2$, etc.
\end{owndescription}

\subsection{Neighborhood}
\label{sec:neighborhood}

The neighborhood(s) used by the search algorithm are defined by
setting the value of \Var{neigh}. The value is a string consisting of
space-separated keywords and possibly numerical values. Neighborhoods
are defined in a postfix notation. The following patterns are used:

\begin{owndescription}
\item[\Var{changeone}] Push a neighborhood on the stack: one orbit is
  changed to another.
\item[\Var{vchangeone}] Push a neighborhood on the stack: one orbit is
  changed to another with equal volume.
\item[\Val{ncount} \Var{addmany}] Push a neighborhood on the
  stack: \Val{ncount} orbits are added.
\item[\Val{ncount} \Var{addmanyunique}] Push a neighborhood on the
  stack: \Val{ncount} orbits (with no repetitions) are added.
\item[\Var{addone}] Push a neighborhood on the stack: one orbit is added.
\item[\Val{ncount} \Var{removemany}] Push a neighborhood on the
  stack: \Val{ncount} orbits are removed.
\item[\Var{removeone}] Push a neighborhood on the stack: remove one orbit.
\item[\Val{ncount} \Var{changemany}] Push a neighborhood on the
  stack: \Val{ncount} orbits are changed.
\item[\Val{ncount} \Var{changemanyunique}] Push a neighborhood on
  the stack: \Val{ncount} patches are changed (no repetions among the
  new patches).
\item[\Var{empty}] Push an empty (no moves) neighborhood.
\item[\Val{n} \Var{composite}] Pop \Val{n} neighborhoods off the stack
  and push a neighborhood where each move is made by appending the moves
  in the \Val{n} subneighborhoods.
\item[\Val{n} \Var{union}] Pop \Var{n} neighborhoods off the stack and
  push a neighborhood whose moves are the union of the \Val{n}
  subneighborhoods.
\item[\Var{balance}] Pop neighborhoods \Var{decN} and \Var{incN}. Push
  neighborhood which finds a too few times covered orbit $a$ of
  $t$-set and a too many times covered orbit ($b$). Push a
  neighborhood that selects a move in \Var{incN} that increases the
  covererage of orbit $a$, and a move in \Var{decN} that decreases the
  coverage of orbit $b$ and composes these two moves.
\item[\Var{largebalance}] Pop neighborhoods \Var{decN} and \Var{incN}.
  Push neighborhood which finds too few times covered orbits and too
  many times covered orbits. The neighborhood then selects a move in
  \Var{incN} that increases the covererage of a too few times covered
  orbit, and a move in \Var{decN} that decreases the coverage of a too
  many time covered orbit, and composes these two moves.
\item[\Var{correct}] Pop neighborhoods
  \Var{decN} and \Var{incN}. Push a neighborhood which finds the next
  too few or too many times covered orbit and then tries to increase
  the coverage (by \Var{incN}) or decrease the coverage (by
  \Var{decN}), respectively.
\item[\Var{correctrandom}] \emph{Covering problems:} The same as
  \Var{correct}, except that instead of going cyclically through the
  points, the next too few or too many times covered orbit is selected
  at random.
\item[\Var{largecorrect}] Pop neighborhoods \Var{decN} and \Var{incN}. Push
  neighborhood which allows any moves of \Var{decN} that decrease the
  coverage of an orbit covered too many times, or moves of \Var{incN}
  that increase the coverage of an orbit covered too few times.
\item[\Val{n} \Val{goal} \Var{select}] Pop \Var{n}
  neighborhoods. Push a neighborhood that works as follows: before
  each major iteration of the search algorithm, check which of the
  \Var{n} neighborhoods has the size closest to \Var{goal}. Use then
  this neighborhood until the next major iteration.
\end{owndescription}

After the value string of \Var{neigh} has been parsed, there should be
a number of neighborhoods (usually 1) on the stack. These
neighborhood(s) are used in the search.

\subsection{Examples}
\label{sec:examples}

\end{document}

%%% Local Variables: 
%%% mode: latex
%%% TeX-master: t
%%% End: 
