% d.tex
%
% Anton Betten 
%
% September 1998
%
% Bayreuth
%
%

\documentclass[10pt,a4paper]{book} 



%
%\usepackage[german]{babel}
%\usepackage[dvips]{epsfig}
%\usepackage{cours11, cours}
%\usepackage{fancyheadings}
%\usepackage{amstex}
\usepackage{amsmath}
\usepackage{amssymb}
\usepackage{latexsym}
\usepackage{epsf}
%\usepackage{supertabular}
%\usepackage{wrapfig}
%\usepackage{blackbrd}
\usepackage{epic,eepic}
\usepackage{rotating}
\usepackage{multicol}
\usepackage{multirow}
\usepackage{makeidx} % additional command see
\usepackage{epsfig}
%\usepackage{amsmath,amsfonts} 

%\usepackage{bibgerm}

\usepackage{mathtime}
%\usepackage{bm}
\usepackage{times}
%\usepackage{pifont}


\input macro.tex

\makeindex

\begin{document} 

{\allowdisplaybreaks%


\bibliographystyle{plain}
%\bibliographystyle{gerplain}
%\makeindex

\renewcommand{\labelenumi}{(\roman{enumi})}

\title{\vspace*{-15mm}
DISCRETA\\
programming manual\\
%\vspace*{1cm}
%\epsfxsize=70mm
%\epsffile{MP/PG_3_2/PG_3_2_8.1}
}
\author{
Anton Betten \\
Fakult\"at f\"ur Mathematik und Physik\\
Universit\"at Bayreuth\\ 
D-95440 Bayreuth\\
{\tt Anton.Betten\symbol{64}uni-bayreuth.de} 
}%end author

%\date{}


\pagenumbering{roman}
\maketitle

\thispagestyle{empty}\phantom{Seite2}\clearpage%
%\addcontentsline{toc}{chapter}{Inhaltsverzeichnis}

\tableofcontents
%\listofsymbols

%\chapter{Introduction}

%\begin{abstract}
%\end{abstract}

%\begin{figure}
%\begin{center}
%\epsfxsize=11cm
%\epsffile{./g_t2_k3.ps}
%\end{center}
%\caption{\label{g4lattice}The graphs on 4 points as a poset of orbits}
%\end{figure}

\chapter{Basic Routines, Files, I/O etc.}
\pagenumbering{arabic}

\input ../obj/TEX/discreta.h.tex
\input ../obj/TEX/discreta.C.tex
\input ../obj/TEX/os.C.tex
\input ../obj/TEX/iof.C.tex
\input ../obj/TEX/nu.C.tex


\chapter{Some Scalar Data Types}
\input ../obj/TEX/in1.C.tex

\chapter{Permutations and Partitions}
\input ../obj/TEX/perm.C.tex
\input ../obj/TEX/perm2.C.tex
\input ../obj/TEX/part.C.tex

\chapter{Vectors, Lists and Matrices}
\input ../obj/TEX/vec.C.tex
\input ../obj/TEX/ma.C.tex

\chapter{Permutation Groups}
\input ../obj/TEX/perm_grp.C.tex
\input ../obj/TEX/perm_grp2.C.tex
\input ../obj/TEX/fga.C.tex
\input ../obj/TEX/lb.C.tex
\input ../obj/TEX/lb_base_change.C.tex

\chapter{Polynomials}
\input ../obj/TEX/poly.C.tex
\input ../obj/TEX/mon.C.tex
\input ../obj/TEX/unip.C.tex

\chapter{Finite Fields}

\input ../obj/TEX/gfq.h.tex
\input ../obj/TEX/gfq_zech.C.tex
\input ../obj/TEX/gfq_nb.C.tex
\input ../obj/TEX/singer.C.tex

\chapter{Linear Groups}

\input ../obj/TEX/perm_rep.C.tex
\input ../obj/TEX/gfq_sz.C.tex
\input ../obj/TEX/gfq_psu.C.tex


%\input ../obj/TEX/kontext.C.tex

\chapter{Linear Codes}
\input ../obj/TEX/bch.C.tex
\input ../obj/TEX/mindist.C.tex

%\chapter{Databases}


\chapter{Double Cosets and Designs}
\input ../obj/TEX/ladder_info.C.tex
\input ../obj/TEX/ladder.C.tex
\input ../obj/TEX/ladder2.C.tex
\input ../obj/TEX/kramer_mesner.C.tex
\input ../obj/TEX/plesken.C.tex
\input ../obj/TEX/intersection.C.tex
\input ../obj/TEX/intersection_aijk.C.tex
\input ../obj/TEX/parameter.C.tex
\input ../obj/TEX/report.C.tex
\input ../obj/TEX/graphical.C.tex
\input ../obj/TEX/mathieu.C.tex
\input ../obj/TEX/dc_draw.C.tex
\input ../obj/TEX/solid.C.tex

%\chapter{Solvable Groups}
%\input ../obj/TEX/fg.C.tex
%\input ../obj/TEX/fg_table.C.tex
%\input ../obj/TEX/fg_iso.C.tex
%\input ../obj/TEX/fg_syl.C.tex
%\input ../obj/TEX/fg_ext.C.tex
%\input ../obj/TEX/fg_color.C.tex
%\input ../obj/TEX/fg_direct.C.tex
%\input ../obj/TEX/aut.C.tex
%\input ../obj/TEX/aut_util.C.tex
%\input ../obj/TEX/aut_init.C.tex
%\input ../obj/TEX/aut_grid.C.tex
%\input ../obj/TEX/aut_dimino.C.tex
%\input ../obj/TEX/cf.C.tex
%\input ../obj/TEX/cl_rep.C.tex
%%\input ../obj/TEX/cl_rep2.C.tex
%\input ../obj/TEX/gt_canon.C.tex
%\input ../obj/TEX/gt_canon_dimino.C.tex
%\input ../obj/TEX/gt_canon_grid.C.tex
%\input ../obj/TEX/gt_col_util.C.tex
%\input ../obj/TEX/gt_color.C.tex


\chapter{Finite Incidence Geometries}
\input ../obj/TEX/ma_geo.C.tex

\chapter{Graphics}
\input ../obj/TEX/vdi.C.tex



%\bibliographystyle{gerplain}% wird oben eingestellt
\bibliography{../../TEX.99/MY_BIBLIOGRAPHY/anton}
% ACHTUNG: nicht vergessen:
% die Zeile
%\addcontentsline{toc}{chapter}{Literaturverzeichnis}
% muss per Hand in d.bbl eingefuegt werden !
% nach \begin{thebibliography}{100}


%\addcontentsline{toc}{chapter}{Index}
%\input{d.ind}


}% allowdisplaybreaks

\end{document}


